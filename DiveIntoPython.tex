\documentclass[a4paper,french,11pt,twoside]{book}

\usepackage[french]{babel}
\usepackage[T1]{fontenc}
\usepackage[utf8]{inputenc} % encodage en utf8
\usepackage[hyphens]{url} % permet l'utilisation des urls avec \url{}
\usepackage[pdfborder=0]{hyperref} % permet les hyperliens dans la table des matières et les footnotes.
\usepackage{fourier}
\usepackage{fullpage}

\sloppy % évite le dépassement des urls dans les marges

\usepackage{color, listings, bera}
\definecolor{keywords}{RGB}{255,0,90}
\definecolor{comments}{RGB}{60,179,113}
\lstset{
language=Python,
xleftmargin=1cm, % indentation du bloc de code
%numbers=left, numberstyle=\scriptsize,
basicstyle=\small\ttfamily,
keywordstyle=\color{keywords},
commentstyle=\color{comments}\emph,
breaklines=true, breakatwhitespace=true, % retour à la ligne activé
postbreak=\raisebox{0ex}[0ex][0ex]{\ensuremath{\hookrightarrow\space}}
}
\lstdefinestyle{none}{
xleftmargin=0cm,
language={},
commentstyle={}
}

\newcounter{example}
\newenvironment{example}[1][]{
  \refstepcounter{example}\par\medskip\noindent
  \textbf{Exemple~\thechapter.\theexample. #1}
  \rmfamily
}{\medskip}

\newcommand\note[2]{\medskip\noindent \rule{1ex}{1ex} \textbf{Note : #1} \\ #2\medskip}
\newcommand\attention[2]{\medskip\noindent \textcolor{red}{\textcircled{!}} \textbf{#1} \\ #2\medskip}



%%%%%%%%%%%%%%%%%%%%%%%%%%%%%%%%%%%%%%%%%%%%%%%%%%%
\hypersetup{ % Modifiez la valeur des champs suivants
    pdfauthor   = {Internet},%Auteurs
    pdftitle    = {Dive into Python - version Française},%Titre du document
    pdfsubject  = {Plongez au coeur de Python},%Sujet
    pdfkeywords = {python},%
    pdfproducer = {PDFLaTeX}}
%%%%%%%%%%%%%%%%%%%%%%%%%%%%%%%%%%%%%%%%%%%%%%%%%%%


\begin{document}

\noindent {\Huge Plongez au coeur de Python}

\bigskip
\noindent \textbf{Première version} : 11 février 2006

\medskip
\noindent Copyright (c) 2000, 2001, 2002, 2003, 2004 Mark \textsc{Pilgrim} \\ (\href{mailto:mark@diveintopython.org}{\textit{mark@diveintopython.org}})

\medskip
\noindent Copyright (c) 2001 Xavier \textsc{Defrang}\\(\href{mailto:xavier@defrang.com}{\textit{xavier@defrang.com}})

\medskip
\noindent Copyright (c) 2004 Jean-Pierre \textsc{Gay}\\(\href{mailto:python@kantoche.org}{\textit{python@kantoche.org}})

\medskip
\noindent Copyright (c) 2004, 2006 Alexandre \textsc{Drahon}\\(\href{mailto:python@adrahon.org}{\textit{python@adrahon.org}})

\medskip
Les évolutions de cet ouvrage (et de sa traduction française) sont disponibles sur le site \url{http://diveintopython.org/}. Si vous le lisez ailleurs, il est possible que vous ne disposiez pas de la dernière version.

\medskip
\noindent Permission vous est donnée de copier, distribuer et/ou modifier ce document selon les termes de la Licence GNU Free Documentation License, Version 1.1 ou ultérieure publiée par la Free Software Foundation, sans Sections Invariables, ni Textes de Première de Couverture, ni Textes de Quatrième de Couverture. Une copie de la licence est incluse en Annexe G, GNU Free Documentation License.

\medskip
\noindent Les programmes d'exemple de ce livre sont des logiciels libres, vous pouvez les redistribuer et/ou les modifier selon les termes de la licence Python publiée par la Python Software Foundation. Une copie de la licence est incluse en Annexe H, Python license.

\bigskip
\noindent \textbf{Version ~\LaTeX~}: Skhaen, quota\_atypique, Cynddl \\
\href{mailto:skhaen@gmail.com}{\textit{skhaen@gmail.com}}

%%%%%%%%%%%%%%%%%%%%%%%%%%%%%%%%%%%%%%%%%%%%%%%%%%%
%%%%%%%%%%%%%%%%%%%%%%%%%%%%%%%%%%%%%%%%%%%%%%%%%%%
\newpage
\tableofcontents
\newpage

%CHAPITRES A INCLURE
\chapter{Installation de Python}\label{Installation de Python}

Bienvenue dans le monde de Python ! Préparez-vous à plonger. Dans ce chapitre, nous allons installer une version de Python appropriée à votre situation.

%%%%%%%%%%%%%%%%%%%%%%%%%%%%%%%%%%%%%%%%%%%%%%%%%
\section{Quel Python vous faut-il ?} \label{QuelPythonVousFautil}

La première chose à faire avec Python est de l'installer. Mais est-ce nécessaire ?

\medskip
Si vous utilisez un compte sur un serveur hébergé, votre FAI a peut-être déjà installé Python. La plupart des distributions Linux les plus courantes installent Python par défaut. Mac OS X 10.2 et les versions suivantes comprennent une version en ligne de commande de Python, mais vous souhaiterez sans doute installer une version ayant une interface graphique plus typique du Mac.

Windows n'inclut aucune version de Python, mais ne désespérez pas ! Il y a de nombreuses manières d'installer Python sous Windows. Comme vous pouvez le constater, Python supporte un grand nombre de systèmes d'exploitation. La liste complète comprend Windows, Mac OS, Mac OS X et tousles systèmes libres compatibles UNIX comme Linux. Il y a aussi des versions pour Sun Solaris, AS/400, Amiga, OS/2, BeOS et une pléthore d'autres plates-formes dont vous n'avez sans doute jamais entendu parler.

De plus, les programmes Python écrits sur une plate-forme peuvent, en prenant en compte certains détails, être exécutés sur toutes les plates-formes supportées. Par exemple, je développe régulièrement des programmes Python sous Windows pour les déployer ensuite sous Linux.

\medskip
Mais revenons à la question de départ, «~Quel Python vous faut-il~?~» Celle qui fonctionne sur la plate-forme que vous utilisez.

%%%%%%%%%%%%%%%%%%%%%%%%%%%%%%%%%%%%%%%%%%%%%%%%%
\section {Python sous Windows}\label{Python sous Windows}

Sous Windows, il y a plusieurs possibilités pour installer Python. ActiveState propose un programme d'installation pour Windows appelé ActivePython, qui comprend une version complète de Python, un IDE doté d'un éditeur de code prenant en compte Python et quelques extensions spécifiques à Windows pour Python qui permettent d'accéder aux services et aux APIs propres à Windows, ainsi qu'à la Base de registre.

ActivePython est librement téléchargeable, bien qu'il ne soit pas open source.C'est l'IDE que j'ai utilisé pour apprendre Python et je vous recommande de l'essayer, à moins que vous n'ayez une raison spécifique de ne pas le faire. Une de ces raisons pourrait être qu'ActiveState a généralement plusieurs mois de retard quand une nouvelle version de Python est publiée. Si vous avez absolument besoin de la dernière version de Python et qu'ActivePython a une version de retard, vous pourrez choisir la deuxième option pour installer Python sous Windows.

La deuxième option est le programme d'installation <<officiel>> de Python,distribué par les gens qui développent Python. Il est librement téléchargeable et open source et il installe toujours la version la plus récente de Python.

%%%%%%%%%%%%%%%%%%%%%%%%%%%%%%%%%%%%%%%%%%%%%%%%%
\subsection*{Procédure 1.1. Option 1 : installer ActivePython}

Voici la procédure pour installer ActivePython :

\begin{enumerate}
    \item{Téléchargez ActivePython depuis \url{http://www.activestate.com/ASPN/Downloads/ActivePython/}.}
    \item{Si vous utilisez Windows 95, Windows 98 ou Windows ME, vous aurez également besoin de télécharger et d'installer Windows Installer 2.0 (\url{http://download.microsoft.com/download/WindowsInstaller/Install/2.0/W9XMe/EN-US/InstMsiA.exe}) avant d'installer ActivePython.}
    \item{Double-cliquez sur le programme d'installation ActivePython-2.2.2-224-win32-ix86.msi.}
    \item{Suivez les étapes du programme d'installation.}
    \item{Si vous manquez d'espace disque, vous pouvez sélectionner l'installation personnalisée et déselectionner la documentation, mais je ne vous le recommande pas, à moins que vous ayez vraiment besoin des 14 Mo.}
    \item{Lorsque l'installation est terminée, fermez le programme d'installation et lancez Démarrer > Programmes > ActiveState ActivePython 2.2 > PythonWin IDE.}
\end{enumerate}

Vous verrez quelque chose comme l'écran suivant :

\begin{lstlisting}[style=none]
PythonWin 2.2.2 (#37, Nov 26 2002, 10:24:37) [MSC 32 bit (Intel)] on win32.
Portions Copyright 1994-2001 Mark Hammond (mhammond@skippinet.com.au) -
see 'Help/About PythonWin' for further copyright information.
>>>
\end{lstlisting}

\subsection*{Procédure 1.2. Option 2 : installer Python depuis Python.org (\url{http://www.python.org/})}

\begin{enumerate}
    \item Téléchargez la dernière version du programme d'installation Python pour Windows depuis \url{http://www.python.org/ftp/python/2.3.3/} en sélectionnant le numéro de version le plus haut et en téléchargeant le programme .exe.
    \item  Double-cliquez sur le programme d'installation, Python-2.xxx.yyy.exe. Le nom dépend de la version de Python disponible au moment du téléchargement.
    \item Suivez les étapes du programme d'installation.
    \item  Si vous manquez d'espace disque, vous pouvez déselectionner le fichier d'aide HTMLHelp, les scripts utilitaires (Tools/) et/ou la suite de tests (Lib/test/).
    \item Si vous n'avez pas les droits d'administrateur de la machine, vous pouvez sélectionner «~Advanced Options~», puis choisir Non-Admin Install. Cette option modifie uniquement l'emplacement des entrées de la Base de registre et des raccourcis du menu Démarrer.
    \item Après l'achèvement de l'installation, fermez le programme d'installation et lancez Démarrer > Programmes > Python 2.3 > IDLE (Python GUI). Vous verrez quelque chose comme l'écran suivant :
\end{enumerate}

\begin{lstlisting}[style=none]
Python 2.3.2 (#49, Oct  2 2003, 20:02:00) [MSC v.1200 32 bit (Intel)] on win32
Type "copyright", "credits" or "license()" for more information.
\begin{center}
    ****************************************************************
    Personal firewall software may warn about the connection IDLE makes to its subprocess using this computer's internal loopback \\
    interface.  This connection is not visible on any external interface and no data is sent to or received from the Internet.
    ****************************************************************
\end{center}
IDLE 1.0
>>>
\end{lstlisting}

%%%%%%%%%%%%%%%%%%%%%%%%%%%%%%%%%%%%%%%%%%%%%%%%%
\section{Python sous Mac OS X}\label{Python sous Mac OS X}

Sous Mac OS X, vous avez deux possibilités, garder la version préinstallée ou installer une nouvelle version. Vous préfèrerez sans doute cette dernière solution.

\medskip
Mac OS X 10.2 et les versions ultérieures contiennent une version en ligne de commande de Python. Si vous êtes à l'aise avec la ligne de commande, vous pouvez utiliser cette version pour le premier tiers du livre. Par contre, la version préinstallée ne contient pas d'analyseur pour le XML, vous devrez donc installer la version complète lorsque vous arriverez au chapitre traitant du XML. Installer la dernière version présente également l'avantage d'une interface graphique interactive.

\subsection*{Procédure 1.3. Exécuter la version préinstallée de Python sous Mac OS X}

Pour utiliser la version préinstallée de Python, suivez ces étapes :

\begin {enumerate}
    \item Ouvrez le dossier /Applications.
    \item Ouvrez le dossier Utilitaires.
    \item Double-cliquez sur Terminal pour ouvrir une fenêtre de terminal et accéder à la ligne de commande.
    \item Tapez python sur la ligne de commande.
\end{enumerate}

\medskip
\noindent Essayez :

\begin{lstlisting}[style=none]
Welcome to Darwin!
[localhost:~] you% python
Python 2.2 (#1, 07/14/02, 23:25:09)
[GCC Apple cpp-precomp 6.14] on darwin
Type "help", "copyright", "credits", or "license" for more information.
>>> [press Ctrl+D to get back to the command prompt]
[localhost:~] you%
\end{lstlisting}

\subsection*{Procédure 1.4. Installation de la dernière version de Python sous Mac OS X}

Suivez ces étapes pour télécharger et installer la dernière version de Python :

\begin{enumerate}
    \item{Téléchargez l'image disque MacPython-OSX depuis \url{http://www.cwi.nl/~jack/macpython.html}. Si votre navigateur n'a pas monté l'image, double-cliquez sur MacPython-OSX-2.3-1.dmg pour la monter sur le bureau.}
    \item{Double-cliquer sur le programme d'installation, MacPython-OSX.pkg.}
    \item{Le programme d'installation vous demandera votre nom et mot de passe d'administrateur.}
    \item{Suivez les étapes du programme d'installation.}
    \item{Lorsque l'installation est terminée, fermez le programme d'installation et ouvrez le dossier /Applications.}
    \item{Ouvrez le dossier MacPython-2.3.}
    \item{Double-cliquez sur PythonIDE pour lancer Python.}
\end{enumerate}

\medskip
L'IDE MacPython devrait afficher un écran d'accueil, puis vous amener à l'interface interactive. Si l'interface n'apparaît pas, sélectionnez \emph{Window -> Python Interactive (Cmd-0)}. La fenêtre qui s'ouvre ressemblera à l'écran suivant :

\begin{lstlisting}
Python 2.3 (#2, Jul 30 2003, 11:45:28)
[GCC 3.1 20020420 (prerelease)]
Type "copyright", "credits" or "license" for more information.
MacPython IDE 1.0.1
>>>
\end{lstlisting}

Notez qu'une fois que vous installez la dernière version, la version préinstallée est toujours présente. Si vous exécutez des scripts depuis la ligne de commande, vous devez savoir quelle version de Python vous utilisez.

\begin{example}[Deux versions de Python]
\begin{lstlisting}[style=none]
[localhost:~] you% python
Python 2.2 (#1, 07/14/02, 23:25:09)
[GCC Apple cpp-precomp 6.14] on darwin
Type "help", "copyright", "credits", or "license" for more information.
>>> [press Ctrl+D to get back to the command prompt]
[localhost:~] you% /usr/local/bin/python
Python 2.3 (#2, Jul 30 2003, 11:45:28)
[GCC 3.1 20020420 (prerelease)] on darwin
Type "help", "copyright", "credits", or "license" for more information.
>>> [press Ctrl+D to get back to the command prompt]
[localhost:~] you%
\end{lstlisting}
\end{example}

%%%%%%%%%%%%%%%%%%%%%%%%%%%%%%%%%%%%%%%%%%%%%%%%%
\section{Python sous Mac OS 9}\label{Python sous Mac OS 9}

Mac OS 9 n'est fournit avec aucune version de Python, mais l'installation en est très simple. Suivez ces étapes pour installer Python sous Mac OS 9 :

\begin{enumerate}
    \item{Téléchargez le fichier MacPython23full.bin depuis \url{http://www.cwi.nl/~jack/macpython.html}.}
    \item{Si votre navigateur n'a pas décompressé le fichier automatiquement, double-cliquez MacPython23full.bin pour le décompresser avec Stuffit Expander.}
    \item{Double-cliquez sur le programme d'installation, MacPython23full.}
    \item{Suivez les étapes du programme d'installation.}
    \item{Lorsque l'installation est terminée, fermez le programme d'installation et  ouvrez le dossier /Applications.}
    \item{Ouvrez le dossier MacPython-OS9 2.3.}
    \item{Double-cliquez Python IDE pour lancer Python.}
\end{enumerate}

\medskip
L'IDE MacPython devrait afficher un écran d'accueil, puis vous amener à l'interface interactive. Si l'interface n'apparaît pas, sélectionnez Window > Python Interactive (Cmd-0). La fenêtre qui s'ouvrira ressemblera à l'écran suivant :

\begin{lstlisting}[style=none]
Python 2.3 (#2, Jul 30 2003, 11:45:28)
[GCC 3.1 20020420 (prerelease)]
Type "copyright", "credits" or "license" for more information.
MacPython IDE 1.0.1
>>>
\end{lstlisting}

%%%%%%%%%%%%%%%%%%%%%%%%%%%%%%%%%%%%%%%%%%%%%%%%%
\section{Python sous RedHat Linux}\label{Python sous RedHat Linux}

L'installation sous un système d'exploitation compatible UNIX tel que Linux est simple si vous choisissez l'installation d'un paquetage binaire. Des paquetages binaires précompilés sont disponibles pour les distributions Linux les plus répandues. Vous pouvez également compiler à partir des sources.
Téléchargez le dernier RPM Python en allant sur \url{http://www.python.org/ftp/python/2.3.3/} et en sélectionnant le numéro de version le plus haut, puis en sélectionnant le sous-répertoire rpms/ de cette version. Téléchargez ensuite le RPM ayant le plus haut numéro de version. Vous pouvez l'installer avec la commande rpm, comme ci-dessous :

\begin{example}[Installation sous RedHat Linux 9]
\begin{lstlisting}[style=none]
localhost:~$ su -
Password: [enter your root password]
[root@localhost root]# wget http://python.org/ftp/python/2.3/rpms/redhat-9/python2.3-2.3-5pydotorg.i386.rpm
Resolving python.org... done.
Connecting to python.org[194.109.137.226]:80... connected.
HTTP request sent, awaiting response... 200 OK
Length: 7,495,111 [application/octet-stream]
...
[root@localhost root]# rpm -Uvh python2.3-2.3-5pydotorg.i386.rpm
Preparing...                ########################################### [100%]
   1:python2.3              ########################################### [100%]
[root@localhost root]# python          (1)
Python 2.2.2 (#1, Feb 24 2003, 19:13:11)
[GCC 3.2.2 20030222 (Red Hat Linux 3.2.2-4)] on linux2
Type "help", "copyright", "credits", or "license" for more information.
>>> [press Ctrl+D to exit]
[root@localhost root]# python2.3       (2)
Python 2.3 (#1, Sep 12 2003, 10:53:56)
[GCC 3.2.2 20030222 (Red Hat Linux 3.2.2-5)] on linux2
Type "help", "copyright", "credits", or "license" for more information.
>>> [press Ctrl+D to exit]
[root@localhost root]# which python2.3 (3)
/usr/bin/python2.3
\end{lstlisting}
\end{example}

\begin{enumerate}
    \item{Attention ! Taper simplement python lance l'ancienne version de Python (celle qui était installée par défaut). Ce n'est pas celle que vous voulez.}
    \item{Au moment où j'écris, la version la plus récente s'appelle python2.3. Vous aurez sans doute à changer le chemin à la première ligne du script pour pointer vers une version plus récente.}
    \item{C'est le chemin complet de la version la plus récente de Python que vous venez d'installer. Utilisez-le sur la ligne \#! (la première ligne de chaque script)  pour vous assurez que les scripts utilisent la dernière version de Python et faites bien attention de taper python2.3 pour accéder à l'interface interactive.}
\end{enumerate}

%%%%%%%%%%%%%%%%%%%%%%%%%%%%%%%%%%%%%%%%%%%%%%%%%
\section{Python sous Debian GNU/Linux}\label{Python sous Debian GNU/Linux}

Si vous avez la chance d'utiliser Debian GNU/Linux, vous pouvez installer Python à l'aide de la commande apt.

\begin{example}[Installation sous Debian GNU/Linux]
\begin{lstlisting}[style=none]
localhost:~$ su -
Password: [enter your root password]
localhost:~# apt-get install python
Reading Package Lists... Done
Building Dependency Tree... Done
The following extra packages will be installed:
  python2.3
Suggested packages:
  python-tk python2.3-doc
The following NEW packages will be installed:
  python python2.3
0 upgraded, 2 newly installed, 0 to remove and 3 not upgraded.
Need to get 0B/2880kB of archives.
After unpacking 9351kB of additional disk space will be used.
Do you want to continue? [Y/n] Y
Selecting previously deselected package python2.3.
(Reading database ... 22848 files and directories currently installed.)
Unpacking python2.3 (from .../python2.3_2.3.1-1_i386.deb) ...
Selecting previously deselected package python.
Unpacking python (from .../python_2.3.1-1_all.deb) ...
Setting up python (2.3.1-1) ...
Setting up python2.3 (2.3.1-1) ...
Compiling python modules in /usr/lib/python2.3 ...
Compiling optimized python modules in /usr/lib/python2.3 ...
localhost:~# exit
logout
localhost:~$ python
Python 2.3.1 (#2, Sep 24 2003, 11:39:14)
[GCC 3.3.2 20030908 (Debian prerelease)] on linux2
Type "help", "copyright", "credits" or "license" for more information.
>>> [press Ctrl+D to exit]
\end{lstlisting}
\end{example}

%%%%%%%%%%%%%%%%%%%%%%%%%%%%%%%%%%%%%%%%%%%%%%%%%
\section{Installation de Python à partir du fichier source}\label{Installation de Python à partir du fichier source}

Si vous préférez l'installer à partir du fichier source, vous pouvez télécharger le code source de Python à partir de \url{http://www.python.org/ftp/python/2.3.3/}. Sélectionnez le numéro de version le plus haut de la liste, téléchargez le fichier \emph{.tgz} et appliquez la séquence habituelle \emph{configure}, \emph{make} et \emph{make install}.

\subsection*{Installation à partir du fichier source}

\begin{lstlisting}[style=none]
localhost:~$ su -
Password: [enter your root password]
localhost:~# wget http://www.python.org/ftp/python/2.3/Python-2.3.tgz
Resolving www.python.org... done.
Connecting to www.python.org[194.109.137.226]:80... connected.
HTTP request sent, awaiting response... 200 OK
Length: 8,436,880 [application/x-tar]
...
localhost:~# tar xfz Python-2.3.tgz
localhost:~# cd Python-2.3
localhost:~/Python-2.3# ./configure
checking MACHDEP... linux2
checking EXTRAPLATDIR...
checking for --without-gcc... no
...
localhost:~/Python-2.3# make
gcc -pthread -c -fno-strict-aliasing -DNDEBUG -g -O3 -Wall -Wstrict-prototypes
-I. -I./Include  -DPy_BUILD_CORE -o Modules/python.o Modules/python.c
gcc -pthread -c -fno-strict-aliasing -DNDEBUG -g -O3 -Wall -Wstrict-prototypes
-I. -I./Include  -DPy_BUILD_CORE -o Parser/acceler.o Parser/acceler.c
gcc -pthread -c -fno-strict-aliasing -DNDEBUG -g -O3 -Wall -Wstrict-prototypes
-I. -I./Include  -DPy_BUILD_CORE -o Parser/grammar1.o Parser/grammar1.c
...
localhost:~/Python-2.3# make install
/usr/bin/install -c python /usr/local/bin/python2.3
...
localhost:~/Python-2.3# exit
logout
localhost:~$ which python
/usr/local/bin/python
localhost:~$ python
Python 2.3.1 (#2, Sep 24 2003, 11:39:14)
[GCC 3.3.2 20030908 (Debian prerelease)] on linux2
Type "help", "copyright", "credits" or "license" for more information.
>>> [press Ctrl+D to get back to the command prompt]
localhost:~$
\end{lstlisting}

%%%%%%%%%%%%%%%%%%%%%%%%%%%%%%%%%%%%%%%%%%%%%%%%%
\section{L'interface interactive}\label{L'interface interactive}

Maintenant que Python est installé, nous allons voir en quoi consiste cette interface interactive que vous avez lancée.
Il faut comprendre une chose : Python mène une double vie. C'est un interpréteur de scripts que vous pouvez lancer depuis la ligne de commande, ou en double-cliquant sur un script. Mais c'est aussi une interface interactive qui peut évaluer n'importe quelle instruction ou expression. C'est très utile pour le débogage, pour écrire rapidement du code et pour les tests. Je connais même des gens qui utilisent l'interface interactive de Python comme calculatrice !

Lancez l'interface interactive de Python de la manière qui convient à la plate-forme et commençons la plongée par les étapes suivantes :

\begin{example}[Premiers pas dans l'interface interactive]

\begin{lstlisting}
>>> 1 + 1               (1)
2
>>> print 'hello world' (2)
hello world
>>> x = 1               (3)
>>> y = 2
>>> x + y
3
\end{lstlisting}
\end{example}

\begin{enumerate}
    \item{L'interface interactive de Python peut évaluer une expression Python quelconque, y compris une expression arithmétique de base.}
    \item{L'interface interactive peut exécuter des instructions Python, y compris l'instruction print.}
    \item{Vous pouvez aussi assigner des valeurs à des variables et les valeurs seront conservées aussi longtemps que l'interface est ouverte (mais pas plus longtemps que cela).}
\end{enumerate}

%%%%%%%%%%%%%%%%%%%%%%%%%%%%%%%%%%%%%%%%%%%%%%%%%
\section{Résumé}\label{Résumé}

Vous devez maintenant avoir une version de Python installée et fonctionnelle.

\medskip
En fonction de votre plate-forme, vous pouvez avoir plus d'une version de Python installée. Si c'est le cas, vous devez connaître vos chemins d'accès. Si entrer simplement  «~python~»  sur la ligne de commande ne lance pas la version de Python que vous souhaitez utiliser, vous aurez peut être à entrer le chemin completde votre version préférée.

\medskip
Félicitations et bienvenue dans le monde de Python !


\chapter{Votre premier programme Python}
% chapitre 2

La plupart des autres livres expliquent pas à pas les concepts de programmation pour vous amener à la fin à l'écriture d'un programme complet et fonctionnel. Et bien nous allons sauter toutes ces étapes.

\section{Plonger dans Python}

Voici un programme Python complet et fonctionnel. Il ne signifie probablement rien pour vous. Ne vous en faites pas, nous allons le disséquer ligne par ligne. Mais lisez-le et voyez ce que vous pouvez déjà en retirer.

\paragraph*{Exemple 2.1. odbchelper.py}

Si vous ne l’avez pas déjà fait, vous pouvez télécharger\footnote{\url{http://diveintopython.org/download/diveintopython-examples-5.4.zip}} cet exemple ainsi que les autres exemples du livre.

\begin{lstlisting}
def buildConnectionString(params):
    """Build a connection string from a dictionary of parameters.
    Returns string."""
    return ";".join(["%s=%s" % (k, v) for k, v in params.items()])
if name == "main":
    myParams = {"server":"mpilgrim", \
                "database":"master", \
                "uid":"sa", \
                "pwd":"secret" \
                }
    print buildConnectionString(myParams)
\end{lstlisting}

Lancez maintenant ce programme et observez ce qui se passe.

\paragraph*{Exécuter un programme sous Windows }

Dans l'IDE ActivePython sous Windows, vous pouvez exécuter le programme Python que vous êtes en train d'éditer par \emph{File} $\rightarrow$ Run…~(Ctrl-R). La sortie est affichée dans la fenêtre interactive.

\paragraph*{Exécuter un programme sous Mac OS}

Dans l'IDE Python sous Mac OS, vous pouvez exécuter un module avec \emph{Python}$\rightarrow$ \emph{Run window…~(Cmd-R)} mais il y a une option importante que vous devez activer préalablement. Ouvrez le module dans l'IDE, ouvrez le menu des options des modules en cliquant le triangle noir dans le coin supérieur droit de la fenêtre et assurez-vous que «~\emph{Run as main}~» est coché. Ce réglage est sauvegardé avec le module, vous n'avez donc à faire cette manipulation qu'une fois par module.

\paragraph*{Exécuter un programe sous Unix}

Sur les systèmes compatibles Unix (y compris Mac OS X), vous pouvez exécuter un programme Python depuis la ligne de commande : \emph{python odbchelper.py}. La sortie de \emph{odbchelper.py} ressemblera à l'écran suivant :

\begin{lstlisting}
server=mpilgrim;uid=sa;database=master;pwd=secret
\end{lstlisting}

\section{Déclaration de fonctions}

Python dispose de fonctions comme la plupart des autre langages, mais il n'a pas de fichiers d'en-tête séparés comme C++ ou de sections interface/implementation comme Pascal. Lorsque vous avez besoin d'une fonction, vous n'avez qu'à la déclarer et l'écrire.

\begin{lstlisting}
def buildConnectionString(params):
\end{lstlisting}

Il y a plusieurs remarques à faire. Premièrement, le mot clé \emph{def} débute une déclaration de fonction, suivi du nom de la fonction, puis des arguments entre parenthèses. Les arguments multiples (non montrés ici) sont séparés par des virgules.

Deuxièmement, la fonction ne définit pas le type de données qu'elle retourne.Les fonctions Python ne définissent pas le type de leur valeur de retour, elle ne spécifient même pas si elle retournent une valeur ou pas. En fait, chaque fonction Python retourne une valeur, si la fonction exécute une instruction \emph{return}, elle va en retourner la valeur, sinon elle retournera \emph{None}, la valeur nulle en Python.

\medskip
\noindent \textbf{Note}: Python vs. Visual Basic : les valeurs de retour \\
En Visual Basic, les fonctions (qui retournent une valeur) débutent avec\emph{function} et les sous-routines (qui ne retournent aucune valeur) débutent avec \emph{sub}. Il n'y a pas de sous-routines en Python. Tout est fonction, toute fonction retourne un valeur (même si c'est \emph{None}) et toute fonction débute avec \emph{def}.

\medskip
Troisièmement, les arguments, \emph{params}, ne spécifient pas de types de données. En Python, les variables ne sont jamais explicitement typées. Python détermine le type d'une variable et en garde la trace en interne.

\medskip
\noindent \textbf{Note}: Python vs. Java : les valeurs de retour\\
En Java, C++ et autres langage à typage statique, vous devez spécifier les types de données de la valeur de retour d'une fonction ainsi que de chaque paramètre. En Python, vous ne spécifiez jamais de manière explicite le type de quoi que ce soit. En se basant sur la valeur que vous lui assignez, Python gère les types de données en interne.

\subsection{Comparaison des types de données en Python et dans d'autres langages de programmation}

Un lecteur érudit propose les explications suivantes pour comparer Python et les autres langages de programmation.

\paragraph*{Langage à typage statique}
Un langage dans lequel les types sont fixés à la compilation. La plupart des langages à typage statique obtiennent cela en exigeant la déclaration de toutes les variables et de leur type avant leur utilisation. Java et C sont des langages à typage statique.

\paragraph*{Langage à typage dynamique}
Un langage dans lequel les types sont découverts à l'exécution, l'inverse du typage statique. VBScript et Python sont des langages à typage dynamique, ils déterminent le type d'une variable la première fois que vous lui assignez une valeur.

\paragraph{Langage fortement typé}
Un langage dans lequel les types sont toujours appliqués. Java et Python sont fortement typés. Un entier ne peut être traité comme une chaîne sans conversion explicite

\paragraph*{Langage faiblement typé}
Un langage dans lequel les types peuvent être ignorés, l'inverse de fortement typé. VBScript est faiblement typé. En VBScript, vous pouvez concaténer la chaîne '12' et l'entier 3 pour obtenir la chaîne '123' et  traiter le résultat comme l'entier 123, le tout sans faire de conversion explicite.

\medskip
Python est donc à la fois à typage dynamique (il n'utilise pas de déclaration de type explicite) et fortement typé (une fois qu'une variable a un type, cela a une importance).

\section*{Documentation des fonctions}
Vous pouvez documenter une fonction Python en lui donnant une chaîne de documentation (\emph{doc string}).

\paragraph*{Exemple 2.2. Définition d'une doc string pour la fonction buildConnectionString}

\begin{lstlisting}
def buildConnectionString(params):
    """Build a connection string from a dictionary of parameters.
    Returns string."""
\end{lstlisting}

Les tripes guillemets indiquent une chaîne multi-lignes. Tout ce qu'il y a entre l'ouverture et la fermeture des guillemets fait partie de la chaîne, y compris les retours chariot et les autres guillemets. On peut les utiliser partout, mais vous les verrez le plus souvent utilisées pour définir une \emph{doc string}.

\medskip
\noindent \textbf{Note} : Python vs. Perl : guillemets

Les triples guillemets sont aussi un moyen simple de définir une chaîne contenant à la fois des guillemets simples et doubles, comme \emph{qq/.../} en Perl.
\medskip

Tout ce qui se trouve entre les triples guillemets fait partie de la doc \emph{string} de la fonction, qui décrit ce que fait la fonction. Une \emph{doc string}, si elle existe, doit être la première chose déclarée dans une fonction (la première chose après les deux points). Techniquement parlant, vous n'êtes pas obligés de donner une \emph{doc string} à votre fonction, mais vous devriez toujours le faire. Je sais que vous avez déjà entendu cela à tous les cours de programmation auxquels vous avez assisté, mais Python vous donne une motivation supplémentaire : la \emph{doc string} est disponible à l'exécution en tant qu'attribut de fonction.

\medskip
\noindent \textbf{Note}: Pourquoi les doc string sont une bonne Chose.\\
Beaucoup d'IDE Python utilisent les \emph{doc string} pour fournir une documentation contextuelle, ainsi lorsque vous tapez le nom d'une fonction, sa \emph{doc string} apparaît dans une bulle d'aide. Cela peut être incroyablement utile, mais cette utilité est liée à la qualité de votre doc string.

\medskip
\noindent Pour en savoir plus sur la documentation des fonctions :
\begin{itemize}
    \item{La PEP 257\footnote{\url{http://www.python.org/peps/pep-0257.html}} définit les conventions pour les doc string.}
    \item{Le Python Style Guide\footnote{\url{http://www.python.org/doc/essays/styleguide.html}} explique la manière d'écrire de bonnes doc string.}
    \item{Le Python Tutorial\footnote{\url{http://www.python.org/doc/current/tut/tut.html}} traite des conventions d'espacement dans les doc string \footnote{\url{http://www.python.org/doc/current/tut/node6.html\#SECTION006750000000000000000}}.}
\end{itemize}

\section{Tout est objet}
Au cas ou vous ne l'auriez pas noté, je viens d'expliquer que les fonctions Python ont des attributs et que ces attributs étaient disponibles au moment de l'exécution. 
Une fonction, comme tout le reste en Python, est un objet. Ouvrez votre IDE Python favorite et suivez ces étapes :

\paragraph*{Exemple 2.3. Accéder à la doc \emph{string} de la fonction \emph{buildConnectionString}}

\begin{lstlisting}
>>> import odbchelper                                   (1)
>>> params = {"server":"mpilgrim","database":"master","uid":"sa","pwd":"secret"}
>>> print odbchelper.buildConnectionString(params)      (2)
server=mpilgrim;uid=sa;database=master;pwd=secret
>>> print odbchelper.buildConnectionString.doc      (3)
Build a connection string from a dictionary
Returns string.
\end{lstlisting}

\begin{enumerate}
\item{La première ligne importe le programme \emph{odbchelper} comme module -- un morceau de code qui peut être utilisé interactivement ou depuis un programme Python (vous verrez des exemples de programmes Python multimodules au Chapitre 4). Une fois que vous importez un module, vous pouvez référencer chacune de ses fonctions, classes ou attributs publics.
Les modules peuvent faire cela pour accéder aux fonctionnalités offertes par d'autres modules et vous pouvez le faire dans l'IDE également. C'est un concept important et nous allons en discuter plus amplement plus tard.}
\item{Quand vous souhaitez utiliser des fonctions définies dans un module importé, vous devez inclure le nom du module. Vous ne pouvez donc pas dire \emph{buildConnectionString}, ce doit être \emph{odbchelper.buildConnectionString}. Si vous avez utilisé des classes en Java, cela devrait vous sembler vaguement familier.}
\item{Plutôt que d'appeler la fonction comme vous l'auriez attendu, nous demandons un des attributs de la fonction, \emph{doc.}}
\end{enumerate}

\paragraph*{Note: Python vs. Perl : import}
\emph{import} en Python est similaire à require en Perl. Une fois que vous importez un module Python, vous accédez à ses fonctions avec \emph{module.function}. Une fois que vous incluez un module Perl, vous accédez à ses fonctions avec \emph{module::function}.

\subsection{Le chemin de recherche d'import}
Avant d'aller plus loin, je veux mentionner rapidement le chemin de recherche de bibliothèques. Python cherche dans plusieurs endroits lorsque vous essayez d'importer un module. Plus précisément, il regarde dans tous les répertoires définis dans \emph{sys.path}. C'est une simple liste et vous pouvez facilement la voir ou la modifier à l'aide des méthodes standard de listes (nous en apprendrons d'avantage sur les listes plus loin dans ce chapitre).

\paragraph{Exemple 2.4. Chemin de recherche d'import}

\begin{lstlisting}
>>> import sys                          (1)
>>> sys.path                            (2)
['', '/usr/local/lib/python2.2', '/usr/local/lib/python2.2/plat-linux2',
'/usr/local/lib/python2.2/lib-dynload', '/usr/local/lib/python2.2/site-packages',
'/usr/local/lib/python2.2/site-packages/PIL', '/usr/local/lib/python2.2/site-packages
/piddle']
>>> sys                                 (3)
<module 'sys' (built-in)>
>>> sys.path.append('/my/new/path')     (4)
\end{lstlisting}

\begin{enumerate}
\item{Importer le module \emph{sys} rend toutes ses fonctions et attributs disponibles.}
\item{\emph{sys.path} est une liste de répertoires qui constitue le chemin de recherche actuel (le votre sera différent en fonction de votre système d'exploitation, la version de Python que vous utilisez et l'endroit où vous l'avez installé). Python recherchera dans ces repertoires (dans l'ordre donné) un fichier \emph{.py} portant le nom de module que vous tentez d'importer.}
\item{En fait j'ai menti, la réalité est plus compliquée que ça car tous les modules ne sont pas dans des fichiers \emph{.py}. Certains, comme le module sys, sont des modules intégrés, il sont inclus dans Python lui-même. Les modules intégrés se comportent comme des modules ordinaires, mais leur code source Python n'est pas disponible car il ne sont pas écrits en Python (le module \emph{sys} est écrit en C).}
\item{Vous pouvez ajouter un nouveau répertoire au chemin de recherche de Python en le joignant à \emph{sys.path} et Python cherchera dans ce répertoire également lorsque vous essayez d'importer un module. Cela dure tant que Python tourne (nous reparlerons de \emph{append} (joindre) et des autres méthodes de listes au Chapitre 3).}
\end{enumerate}

\subsection{Qu'est-ce qu'un objet ?}

En Python, tout est objet et presque tout dispose d'attributs et de méthodes. Toutes les fonctions ont un attribut prédéfini \emph{doc} qui retourne la doc \emph{string} définie dans le code source de la fonction. Le module \emph{sys} est un objet qui a (entre autres choses) un attribut appelé \emph{path}. Et ainsi de suite.

Reste la question : qu'est-ce qu'un objet ? Chaque langage de programmation définit le terme «~\emph{objet}~» à sa manière. Pour certain, cela signifie que tout objet doit avoir des attributs et des méthodes, pour d'autres, cela signifie que tout les objets doivent être dérivables. En Python, la définition est plus flexible. Certains objets n'ont ni attributs ni méthodes (nous verrons cela au Chapitre 3) et tous les objets ne sont pas dérivables (voir le Chapitre 5). Mais tout est objet dans le sens où tout peut être assigné à une variable ou passé comme argument à une fonction (voir au Chapitre 4).

\medskip
Ceci est important et il ne fait aucun mal de le souligner une derniere fois : en Python tout est objet. Les chaînes sont des objets. Les listes sont des objets. Les fonctions sont des objets. Même les modules sont des objets.

\medskip
\noindent Pour en savoir plus sur les objets

\medskip
\begin{itemize}
\item{La Python Reference Manual\footnote{\url{http://www.python.org/doc/current/ref/}} explique précisémment ce qu'implique de dire que tout est objet en Python\footnote{\url{http://www.python.org/doc/current/ref/objects.html}}, puisque certains pédants aiment discuter longuement de ce genre de choses.} 
\item{eff-bot\footnote{\url{http://www.effbot.org/guides/}} propose un résumé des objets Python\footnote{\url{http://www.effbot.org/guides/python-objects.htm}}.}
\end{itemize}

\section{Indentation du code}

Les fonctions Python n'ont pas de begin ou end explicites, ni d'accolades qui pourraient marquer là ou commence et ou se termine le code de la fonction. Le seul délimiteur est les deux points (<<:>>) et l'indentation du code lui-même.

\paragraph*{Exemple 2.5. Indentation de la fonction buildConnectionString}

\begin{lstlisting}
def buildConnectionString(params):
    """Build a connection string from a dictionary of parameters.
    Returns string."""
    return ";".join(["%s=%s" % (k, v) for k, v in params.items()])
\end{lstlisting}

Les blocs de code (fonctions, instructions \emph{if}, boucles \emph{for} ou \emph{while} etc.) sont définis par leur indentation. L'indentation démarre le bloc et la désindendation le termine. Il n'y a pas d'accolades, de crochets ou de mots clés spécifiques. Cela signifie que les espaces blancs sont significatifs et qu'ils doivent être cohérents. Dans cet exemple, le code de la fonction -- y compris sa doc string -- sont indentés de 4 espaces. Cela ne doit pas être forcément 4 espaces, mais il faut que ce soit cohérent. La première ligne non indentée est en dehors de la fonction.
L'exemple 2.6, «~Instructions \emph{if}~» montre un exemple d'indentation du code avec des instructions \emph{if}.

\paragraph*{Exemple 2.6. Instructions if}
\begin{lstlisting}
def fib(n):                   (1)
    print 'n =', n            (2)
    if n > 1:                 (3)
        return n * fib(n - 1)
    else:                     (4)
        print 'end of the line'
        return 1
\end{lstlisting}

\begin{itemize}
    \item{Voici une fonction nommée \emph{fib} qui prend un argument «~n~». Tout le code de cette fonction est indenté.}
    \item{Afficher une sortie à l'écran est très facile en Python, il suffit d'utiliser print. Les instructions print peuvent prendre n'importe quel type de données, y compris les chaînes, les entiers et d'autres types prédéfinis comme les dictionnaires et les listes, que vous découvrirez dans le prochain chapitre. Vous pouvez même mélanger les types pour imprimer plusieurs éléments sur la même ligne en utilisant une liste de valeurs séparées par des virgules. Chaque valeur est affichée sur la même ligne, séparée par des espaces (les virgules ne sont pas imprimées). Donc, lorsque fib est appelé avec 5, cette ligne affichera "n = 5".}
    \item{Les instructions if sont un type de bloc de code. Si l'expression if est évaluée à vrai, le bloc de code indenté est exécuté, sinon on saute au bloc else.}
    \item{Bien sûr, les blocs if et else peuvent contenir des lignes multiples, tant qu'elles sont toutes indentées au même niveau. Ce bloc else contient deux lignes de code. Il n'y a pas d'autre syntaxe pour les blocs de codes multilignes. Indentez et c'est tout.}
\end{itemize}

\medskip
Après quelques protestations initiales et des analogies méprisantes à Fortran, vous vous en accomoderez et commencerez à en voir les bénéfices. Un des bénéfices majeurs est que tous les programmes Python ont la même apparence puisque l'indentation est une caractéristique du langage et non une question de style. Cela rend plus simple à comprendre le code Python des autres.

\paragraph{Note : Python vs. Java : séparation des instructions}
Python utilise le retour chariot pour séparer les instructions, deux points et l'indentation pour séparer les blocs de code. C++ et Java utilisent des points-virgules pour séparer les instructions et des accolades pour séparer les blocs de code.

\medskip
\noindent Pour en savoir plus sur l'indentation du code
\begin{itemize}
    \item La Python Reference Manual\footnote{\url{http://www.python.org/doc/current/ref/}} discute des aspects multiplate-formes de l'indentation et présente diverses erreurs d'indentation (\footnote{\url{http://www.python.org/doc/current/ref/indentation.html}}.
    \item Le Python Style Guide \footnote{\url{http://www.python.org/doc/essays/styleguide.html}} discute du bon usage de l'indentation.
\end{itemize}

\section{Test des modules}

Les modules Python sont des objets et ils ont de nombreux attributs utiles. C'est un aspect que vous pouvez utiliser pour tester facilement vos modules au cours de leur écriture. Voici un exemple qui utilise l'astuce ifname.

\begin{lstlisting}
if name == "main":
\end{lstlisting}

Quelques remarques avant de passer aux choses sérieuses. Premièrement, les parenthèses ne sont pas obligatoires autour de l'expression if. Ensuite, l'instruction if se termine par deux points et est suivie de code indenté.

\paragraph*{Note: Python vs. C: comparaison et assignation}
A l'instar de C, Python utilise == pour la comparaison et = pour l'assignement. Mais au contraire de C, Python ne permet pas les assignations dans le corps d'une instruction afin d'éviter qu'une valeur soit accidentellement assignée alors que vous pensiez effectuer une simple comparaison.
En quoi cette instruction \emph{if} est-elle une astuce ? Les modules sont des objets et tous les modules disposent de l'attribut prédéfini \emph{name}. Le \emph{name} d'un module dépend de la façon dont vous l'utilisez. Si vous importez le module, son \emph{name} est le nom de fichier du module sans le chemin d'accès ni le suffixe. Mais vous pouvez aussi lancer le module directement en tant que programme, dans ce cas \emph{name} va prendre par défaut une valeur spéciale, \emph{main}.

\begin{lstlisting}
>>> import odbchelper
>>> odbchelper.name'odbchelper'
\end{lstlisting}

Sachant cela, vous pouvez concevoir une suite de tests pour votre module au sein même de ce dernier en la plaçant dans ce if. Quand vous lancez le module directement, \emph{name} est \emph{main} et la séquence de tests s'exécute. Quand vous importez le module, \emph{name} est autre chose et les tests sont ignorés. Cela facilite le développement et le déboguage de nouveaux modules avant leur intégration dans un programme plus grand.

\paragraph*{Astuce : if name sous Mac OS}
Avec MacPython, il y a une étape supplémentaire pour que l'astuce \emph{if name} fonctionne. Ouvrez le menu des options des modules en cliquant le triangle noir dans le coin supérieur droit de la fenêtre et assurez-vous que \emph{Run as main} est coché.

\medskip
\noindent Pour en savoir plus sur l'importation des modules

\begin{itemize}
    \item{Python Reference Manual\footnote{\url{http://www.python.org/doc/current/ref/}} (Le Manuel de référence pour Python)}
    \item{Détails techniques de l'importation de modules\footnote{\url{http://www.python.org/doc/current/ref/import.html}}}
\end{itemize}

\chapter{Types prédéfinis}\label{Types prédéfinis}
% Chapitre 3

Avant de revenir à votre premier programme Python, une petite digression est de rigueur, car vous devez absolument connaître les dictionnaires, les tuples et les listes (tout ça !). Si vous être un programmeur Perl, vous pouvez probablement passer rapidement sur les points concernant les dictionnaires et les listes, mais vous devrez quand même faire attention aux tuples.

\section{Présentation des dictionnaires}\label{Présentation des dictionnaires}

Un des types de données fondamentaux de Python est le dictionnaire, qui définit une relation 1 à 1 entre des clés et des valeurs.

\note{Python vs. Perl: les dictionnaires}{
En Python, un dictionnaire est comme une table de hachage en Perl. En Perl, les variables qui stockent des tables de hachage débutent toujours par le caractère \emph{\%}. En Python vous pouvez nommer votre variable comme bon vous semble et Python se chargera de la gestion du typage.}

\note{Python vs. Java: les dictionnaires}{
Un dictionnaire Python est similaire à une instance de la classe \emph{Hashtable} en Java.}

\note{Python vs. Visual Basic: les dictionnaires}{
Un dictionnaire Python est similaire à une instance de l'objet \emph{Scripting.Dictionnary} en Visual Basic.}

\subsection{Définition des dictionnaires}

\begin{example}[Définition d'un dictionnaire]
\begin{lstlisting}
>>> d = {"server":"mpilgrim", "database":"master"} (1)
>>> d
{'server': 'mpilgrim', 'database': 'master'}
>>> d["server"]                                    (2)
'mpilgrim'
>>> d["database"]                                  (3)
'master'
>>> d["mpilgrim"]                                  (4)
Traceback (innermost last):
  File "<interactive input>", line 1, in ?
KeyError: mpilgrim
\end{lstlisting}
\end{example}

\begin{enumerate}
\item{D'abord, nous créons un nouveau dictionnaire avec deux éléments que nous assignons à la variable \emph{d}. Chaque élément est une paire clé-valeur et l'ensemble complet des éléments est entouré d'accolades.}
\item{\emph{'server'} est une clé et sa valeur associée, référencée par \emph{d["server"]}, est \emph{'mpilgrim'}.}
\item{\emph{'database'} est une clé et sa valeur associée, référencée par \emph{d["database"]}, est \emph{'master'}.}
\item{Vous pouvez obtenir les valeurs par clé, mais pas les clés à partir de leur valeur. Donc \emph{d["server"]} est \emph{'mpilgrim'}, mais \emph{d["mpilgrim"]} déclenche une exception, car \emph{'mpilgrim'} n'est pas une clé.}
\end{enumerate}

\subsection{Modification des dictionnaires}

\begin{example}[Modification d'un dictionnaire]
\begin{lstlisting}
>>> d
{'server': 'mpilgrim', 'database': 'master'}
>>> d["database"] = "pubs" (1)
>>> d
{'server': 'mpilgrim', 'database': 'pubs'}
>>> d["uid"] = "sa"        (2)
>>> d
{'server': 'mpilgrim', 'uid': 'sa', 'database': 'pubs'}
\end{lstlisting}
\end{example}

\begin{enumerate}
\item{Vous ne pouvez pas avoir de clés dupliquées dans un dictionnaire. L'assignation d'une valeur à une clé existante a pour effet d'effacer l'ancienne valeur.}
\item{Vous pouvez ajouter de nouvelles paires clé-valeur à tout moment. La syntaxe est identique à celle utilisée pour modifier les valeurs existantes. (Oui, cela vous posera problème si vous essayez d'ajouter de nouvelles valeurs alors que vous ne faites que modifier constamment la même valeur parce que votre clé n'a pas changé de la manière que vous espériez)}
\end{enumerate}

Notez que le nouvel élément (clé \emph{'uid'}, valeur \emph{'sa'}) à l'air d'être au milieu. En fait c'est par coïncidence que les éléments avaient l'air d'être dans l'ordre dans le premier exemple, c'est tout autant une coïncidence qu'ils aient l'air dans le désordre maintenant.

\note{Les dictionnaires ne sont pas ordonnés}{
Les dictionnaires ne sont liés à aucun concept d'ordonnancement des éléments. Il est incorrect de dire que les éléments sont «~dans le désordre~», ils ne sont tout simplement pas ordonnés. C'est une distinction importante qui vous ennuiera lorsque vous souhaiterez accéder aux éléments d'un dictionnaire d'une façon spécifique et reproductible (par exemple par ordre alphabétique des clés). C'est possible, mais cette fonctionnalité n'est pas intégrée au dictionnaire.}

Quand vous utilisez des dictionnaires, vous devez garder à l'esprit le fait que les clés sont sensibles à la casse.

\begin{example}[Les clés des dictionnaires sont sensibles à la casse]
\begin{lstlisting}
>>> d = {}
>>> d["key"] = "value"
>>> d["key"] = "other value" (1)
>>> d
{'key': 'other value'}
>>> d["Key"] = "third value" (2)
>>> d
{'Key': 'third value', 'key': 'other value'}
\end{lstlisting}
\end{example}

\begin{enumerate}
\item{Assigner une valeur a une clé existante remplace l'ancienne valeur par la nouvelle.}
\item{Ici la valeur n'est pas assignée à une clé existante parce que les chaînes en Python sont sensibles à la casse, donc \emph{'key'} n'est pas la même chose que \emph{'Key'}. Une nouvelle paire clé/valeur est donc créée dans le dictionnaire, elle peut vous sembler similaire à la précédente, mais pour Python elle est complètement différente.}
\end{enumerate}

\begin{example}[Mélange de types de données dans un dictionnaire]
\begin{lstlisting}
>>> d
{'server': 'mpilgrim', 'uid': 'sa', 'database': 'pubs'}
>>> d["retrycount"] = 3 (1)
>>> d
{'server': 'mpilgrim', 'uid': 'sa', 'database': 'master', 'retrycount': 3}
>>> d[42] = "douglas"   (2)
>>> d
{'server': 'mpilgrim', 'uid': 'sa', 'database': 'master',
42: 'douglas', 'retrycount': 3}
\end{lstlisting}
\end{example}

\begin{enumerate}
\item{Les dictionnaires ne servent pas uniquement aux chaînes de caractères. Les valeurs d'un dictionnaire peuvent être de n'importe quel type de données, y compris des chaînes, des entiers, des objets et même d'autres dictionnaires. Au sein d'un même dictionnaire, les valeurs ne sont pas forcément d'un même type, vous pouvez les mélanger à votre guise.}
\item{Les clés d'un dictionnaire sont plus restrictives, mais elles peuvent être des chaînes, des entiers et de quelques autres types encore (nous verrons cela en détail plus tard). Vous pouvez également mélanger divers types de données au sein des clés d'un dictionnaire.}
\end{enumerate}

\subsection{Enlever des éléments d'un dictionnaire}

\begin{example}[Enlever des éléments d'un dictionnaire]
\begin{lstlisting}
>>> d
{'server': 'mpilgrim', 'uid': 'sa', 'database': 'master',
42: 'douglas', 'retrycount': 3}
>>> del d[42] (1)
>>> d
{'server': 'mpilgrim', 'uid': 'sa', 'database': 'master', 'retrycount': 3}
>>> d.clear() (2)
>>> d
{}
\end{lstlisting}
\end{example}

\begin{enumerate}
\item{L'instruction \emph{del} vous permet d'effacer des éléments d'un dictionnaire en fonction de leur clé.}
\item{La méthode \emph{clear} efface tous les éléments d'un dictionnaire. Notez que l'ensemble fait d'accolades vides signifie un dictionnaire sans éléments.}
\end{enumerate}

\paragraph{Pour en savoir plus sur les dictionnaires}
\begin{itemize}
\item{\emph{How to Think Like a Computer Scientist}\footnote{\url{http://www.ibiblio.org/obp/thinkCSpy/}} explique comment utiliser les dictionnaires pour modéliser les matrices creuses\footnote{\url{http://www.ibiblio.org/obp/thinkCSpy/chap10.htm}}.}
\item{La \emph{Python Knowledge Base}\footnote{\url{http://www.faqts.com/knowledge-base/index.phtml/fid/199/}} a de nombreux exemples de code ayant recours aux dictionnaires\footnote{\url{http://www.faqts.com/knowledge-base/index.phtml/fid/541}}.}
\item{Le \emph{Python Cookbook}\footnote{\url{http://www.activestate.com/ASPN/Python/Cookbook/}} explique comment trier les valeurs d'un dictionnaire par leurs clés\footnote{\url{http://www.activestate.com/ASPN/Python/Cookbook/Recipe/52306}}.}
\item{La \emph{Python Library Reference}\footnote{\url{http://www.python.org/doc/current/lib/}} résume toutes les méthodes des dictionnaires\footnote{\url{http://www.python.org/doc/current/lib/typesmapping.html}}.}
\end{itemize}

\section{Présentation des listes}\label{Présentation des listes}

Les listes sont le type de données à tout faire de Python. Si votre seule expérience des listes se limite à l'utilisation des tableaux de Visual Basic ou -- à Dieu ne plaise -- les datastores de Powerbuilder, accrochez-vous pour les listes Python.

\note{Python vs. Perl : listes}{
Une liste en Python est comme un tableau Perl. En Perl, les variables qui stockent des tableaux débutent toujours par le caractère \emph{@}, en Python vous pouvez nommer votre variable comme bon vous semble et Python se chargera de la gestion du typage.}

\note{Python vs. Java : listes}{
Une liste Python est bien plus qu'un tableau en Java (même s'il peut être utilisé comme tel si vous n'attendez vraiment rien de mieux de la vie). Une meilleure analogie serait la classe \emph{ArrayList}, qui peut contenir n'importe quels objets et qui croît dynamiquement au fur et à mesure que de nouveaux éléments y sont ajoutés.}

\subsection{Définition d'une liste}

\begin{example}[Definition d'une liste]
\begin{lstlisting}
>>> li = ["a", "b", "mpilgrim", "z", "example"] (1)
>>> li
['a', 'b', 'mpilgrim', 'z', 'example']
>>> li[0]                                       (2)
'a'
>>> li[4]                                       (3)
'example'
\end{lstlisting}
\end{example}

\begin{enumerate}
\item{Premièrement, nous définissons une liste de 5 éléments. Notez qu'ils conservent leur ordre d'origine. Ce n'est pas un accident. Une liste est un ensemble ordonné d'éléments entouré par des crochets.}
\item{Une liste peut être utilisée comme un tableau dont l'indice de base est zéro. Le premier élément de toute liste non vide est toujours \emph{li[0]}.}
\item{Le dernier élément de cette liste de 5 éléments est \emph{li[4]}, car les listes sont toujours indicées à partir de zéro.}
\end{enumerate}

\begin{example}[Indices de liste négatifs]
\begin{lstlisting}
>>> li
['a', 'b', 'mpilgrim', 'z', 'example']
>>> li[-1] (1)
'example'
>>> li[-3] (2)
'mpilgrim'
\end{lstlisting}
\end{example}

\begin{enumerate}
\item{Un indice négatif permet d'accéder aux éléments à partir de la fin de la liste en comptant à rebours. Le dernier élément de toute liste non vide est toujours \emph{li[-1]}.}
\item{Si vous trouvez que les indices négatifs prêtent à confusion, voyez-les comme suit : \emph{li[n] == li[n - len(li)]}. Donc dans cette liste, \emph{li[-3] == li[5 - 3] == li[2]}.}
\end{enumerate}

\begin{example}[Découpage d'une liste]
\begin{lstlisting}
>>> li
['a', 'b', 'mpilgrim', 'z', 'example']
>>> li[1:3]  (1)
['b', 'mpilgrim']
>>> li[1:-1] (2)
['b', 'mpilgrim', 'z']
>>> li[0:3]  (3)
['a', 'b', 'mpilgrim']
\end{lstlisting}
\end{example}

\begin{enumerate}
\item{Vous pouvez obtenir un sous-ensemble d'une liste, appelé une « tranche » (\emph{slice}), en spécifiant deux indices. La valeur de retour est une nouvelle liste contenant les éléments de la liste, dans l'ordre, en démarrant du premier indice de la tranche (dans ce cas \emph{li[1]}), jusqu'à au second indice de la tranche non inclu (ici \emph{li[3]}).}
\item{Le découpage fonctionne si un ou les deux indices sont négatifs. Pour vous aider, vous pouvez les voir comme ceci : en lisant la liste de gauche à droite, le premier indice spécifie le premier élément que vous désirez et le second indice spécifie le premier élément dont vous ne voulez pas. La valeur de retour est tout ce qui se trouve entre les deux.}
\item{Les listes sont indicées à partir de zéro, donc \emph{li[0:3]} retourne les trois premiers éléments de la liste, en démarrant à \emph{li[0]} jusqu'à \emph{li[3]} non inclu.}
\end{enumerate}

\begin{example}[Raccourci pour le découpage]
\begin{lstlisting}
>>> li
['a', 'b', 'mpilgrim', 'z', 'example']
>>> li[:3] (1)
['a', 'b', 'mpilgrim']
>>> li[3:] (2) (3)
['z', 'example']
>>> li[:]  (4)
['a', 'b', 'mpilgrim', 'z', 'example']
\end{lstlisting}
\end{example}

\begin{enumerate}
\item{Si l'indice de tranche de gauche est 0, vous pouvez l'omettre et 0 sera implicite. Donc \emph{li[:3]} est la même chose que \emph{li[0:3]} dans le premier exemple.}
\item{De la même manière, si l'indice de tranche de droite est la longueur de la liste, vous pouvez l'omettre. Donc \emph{li[3:]} est pareil que \emph{li[3:5]}, puisque la liste a 5 éléments.}
\item{Remarquez la symétrie. Dans cette liste de 5 éléments, \emph{li[:3]} retourne les 3 premiers éléments et \emph{li[3:]} retourne les deux derniers. En fait \emph{li[:n]} retournera toujours les n premiers éléments et \emph{li[n:]} le reste, quelle que soit la longueur de la liste.}
\item{Si les deux indices sont omis, tous les éléments de la liste sont inclus dans la tranche. Mais ce n'est pas la même chose que la liste \emph{li} ; c'est une nouvelle liste qui contient les mêmes éléments. \emph{li[:]} est un raccourci permettant d'effectuer une copie complète de la liste.}
\end{enumerate}

\subsection{Ajout d'éléments à une liste}

\begin{example}[Ajout d'éléments à une liste]
\begin{lstlisting}
>>> li
['a', 'b', 'mpilgrim', 'z', 'example']
>>> li.append("new")               (1)
>>> li
['a', 'b', 'mpilgrim', 'z', 'example', 'new']
>>> li.insert(2, "new")            (2)
>>> li
['a', 'b', 'new', 'mpilgrim', 'z', 'example', 'new']
>>> li.extend(["two", "elements"]) (3)
>>> li
['a', 'b', 'new', 'mpilgrim', 'z', 'example', 'new', 'two', 'elements']
\end{lstlisting}
\end{example}

\begin{enumerate}
\item{\emph{append} ajoute un élément à la fin de la liste.}
\item{\emph{insert} insère un élément dans la liste. L'argument numérique est l'indice du premier élément qui sera décalé. Notez que les éléments de la liste ne sont pas obligatoirement uniques ; il y a maintenant 2 éléments distincts avec la valeur \emph{'new'},  \emph{li[2]} and  \emph{li[6]}.}
\item{\emph{extend} concatène des listes. Notez que vous n'appelez pas \emph{extend} avec plusieurs arguments, mais bien avec un seul argument qui est une liste. Dans le cas présent, la liste est composée de deux éléments.}
\end{enumerate}

\begin{example}[Différence entre extend et append]
\begin{lstlisting}
>>> li = ['a', 'b', 'c']
>>> li.extend(['d', 'e', 'f']) (1)
>>> li
['a', 'b', 'c', 'd', 'e', 'f']
>>> len(li)                    (2)
6
>>> li[-1]
'f'
>>> li = ['a', 'b', 'c']
>>> li.append(['d', 'e', 'f']) (3)
>>> li
['a', 'b', 'c', ['d', 'e', 'f']]
>>> len(li)                    (4)
4
>>> li[-1]
['d', 'e', 'f']
\end{lstlisting}
\end{example}

\begin{enumerate}
\item {Les listes ont deux méthodes, \emph{extend} et \emph{append}, qui semblent faire la même chose, mais sont en fait complètement différentes. \emph{extend} prend un seul argument, qui est toujours une liste et ajoute chacun des éléments de cette liste à la liste originelle.}
\item {Ici nous avons une liste de trois éléments (\emph{'a'}, \emph{'b'} et \emph{'c'}) et nous utilisons \emph{extended} pour lui ajouter une liste de trois autres éléments (\emph{'d'}, \emph{'e'} et \emph{'f'}), ce qui nous donne une liste de six éléments.}
\item {Par contre, \emph{append} prend un argument, qui peut être de n'importe quel type et l'ajoute simplement à la fin de la liste. Ici, nous appelons \emph{append} avec un argument, qui est une liste de trois éléments.}
\item {Maintenant, la liste originelle qui avait trois éléments en contient quatre. Pourquoi quatre ? Parce que le dernier élément que nous venons d'ajouter est lui-même une liste. Les listes peuvent contenir n'importe quel type de données, y compris d'autres listes. En fonction du but recherché, faites attention de ne pas utiliser \emph{append} si vous pensez en fait à \emph{extend}.}
\end{enumerate}

\subsection{Recherche dans une liste}

\begin{example}[Recherche dans une liste]
\begin{lstlisting}
>>> li
['a', 'b', 'new', 'mpilgrim', 'z', 'example', 'new', 'two', 'elements']
>>> li.index("example") (1)
5
>>> li.index("new")     (2)
2
>>> li.index("c")       (3)
Traceback (innermost last):
  File "<interactive input>", line 1, in ?
ValueError: list.index(x): x not in list
>>> "c" in li           (4)
False
\end{lstlisting}
\end{example}

\begin{enumerate}
\item{\emph{index} trouve la première occurrence d'une valeur dans la liste et retourne son indice.}
\item{\emph{index} trouve la première occurrence d'une valeur dans la liste. Dans ce cas, \emph{new} apparaît à deux reprises dans la liste, \emph{li[2]} et \emph{li[6]}, mais \emph{index} ne retourne que le premier indice, 2.}
\item{Si la valeur est introuvable dans la liste, Python déclenche une exception.C'est sensiblement différent de la plupart des autres langages qui retournent un indice invalide. Si cela peut sembler gênant, c'est en fait une bonne chose, car cela signifie que votre programme se plantera à la source même du problème plutôt qu'au moment où vous tenterez de manipuler l'indice non valide.}
\item{Pour tester la présence d'une valeur dans la liste, utilisez \emph{in}, qui retourne \emph{True} si la valeur a été trouvée et \emph{False} dans le cas contraire.}
\end{enumerate}

\note{Qu'est-ce qui est vrai en Python ?}{
Avant la version 2.2.1, Python n'avait pas de type booléen. Pour compenser cela, Python acceptait pratiquement n'importe quoi dans un contexte requérant un booléen (comme une instruction \emph{if}), en fonction des règles suivantes :
\begin{itemize}
\item{0 est faux, tous les autres nombres sont vrais ;}
\item{une chaîne vide (\emph{""}) est faux, toutes les autres chaînes sont vrai ;}
\item{une liste vide (\emph{[]}) est faux, toutes les autres listes sont vrai ;}
\item{un tuple vide (\emph{()}) est faux, tous les autres tuples sont vrai ;}
\item{un dictionnaire vide (\emph{\{\}}) est faux, tous les autres dictionnaires sont vrai.}
\end{itemize}   
Ces règles sont toujours valides en Python 2.3.3 et au-delà, mais vous pouvez maintenant utiliser un véritable booléen, qui a pour valeur \emph{True} ou \emph{False}. Notez la majuscule, ces valeurs comme tout le reste en Python, sont sensibles à la casse.}

\subsection{Suppression d'éléments d'une liste}

\begin{example}[Enlever des éléments d'une liste]
\begin{lstlisting}
>>> li
['a', 'b', 'new', 'mpilgrim', 'z', 'example', 'new', 'two', 'elements']
>>> li.remove("z")   (1)
>>> li
['a', 'b', 'new', 'mpilgrim', 'example', 'new', 'two', 'elements']
>>> li.remove("new") (2)
>>> li
['a', 'b', 'mpilgrim', 'example', 'new', 'two', 'elements']
>>> li.remove("c")   (3)
Traceback (innermost last):
  File "<interactive input>", line 1, in ?
ValueError: list.remove(x): x not in list
>>> li.pop()         (4)
'elements'
>>> li
['a', 'b', 'mpilgrim', 'example', 'new', 'two']
\end{lstlisting}
\end{example}

\begin{enumerate}
\item{\emph{remove} enlève la première occurrence de la valeur de la liste.}
\item{\emph{remove} enlève uniquement la première occurence de la valeur. Dans ce cas, new apparaît à deux reprises dans la liste, mais \emph{li.remove("new")} a seulement retiré la première occurrence.}
\item{Si la valeur est introuvable dans la liste, Python déclenche une exception. Ce comportement est identique à celui de la méthode \emph{index}.}
\item{\emph{pop} est un spécimen intéressant. Il fait deux choses : il enlève le dernier élément de la liste et il retourne la valeur qui a été enlevée. Notez que cela diffère de \emph{li[-1]} qui retourne une valeur, mais ne modifie pas la liste  et de \emph{li.remove(valeur)} qui altère la liste, mais ne retourne pas de valeur.}
\end{enumerate}

\subsection{Utilisation des opérateurs de listes}

\begin{example}[Opérateurs de listes]
\begin{lstlisting}
>>> li = ['a', 'b', 'mpilgrim']
>>> li = li + ['example', 'new'] (1)
>>> li
['a', 'b', 'mpilgrim', 'example', 'new']
>>> li += ['two']                (2)
>>> li
['a', 'b', 'mpilgrim', 'example', 'new', 'two']
>>> li = [1, 2] * 3              (3)
>>> li
[1, 2, 1, 2, 1, 2]
\end{lstlisting}
\end{example}

\begin{enumerate}
\item{Les listes peuvent être concaténées à l'aide de l'opérateur \emph{+.} \emph{liste = liste + autreliste} est équivalent à \emph{list.extend(autreliste)}. Mais l'opérateur \emph{+} retourne une nouvelle liste concaténée comme une valeur alors que \emph{extend} modifie une liste existante. Cela implique que \emph{extend} est plus rapide, surtout pour de grandes listes.}
\item{Python supporte l'opérateur \emph{+=. li += ['two']} est équivalent à li = li + ['two']. L'opérateur \emph{+=} fonctionne pour les listes, les chaînes et les entiers. Il peut être surchargé pour fonctionner également avec des classes définies par l'utilisateur (nous en apprendrons plus sur les classes au chapitre 5).}
\item{L'opérateur \emph{*} agit sur les liste comme un répéteur. \emph{li = [1, 2] * 3} est équivalent à \emph{li = [1, 2] + [1, 2] + [1, 2]}, qui concatène les trois listes en une seule.}
\end{enumerate}

\paragraph{Pour en savoir plus sur les listes}
\begin{itemize}
\item{How to Think Like a Computer Scientist\footnote{\url{http://www.ibiblio.org/obp/thinkCSpy/}} explique les listes et expose le sujet important du passage de listes comme arguments de fonction \footnote{\url{http://www.ibiblio.org/obp/thinkCSpy/chap08.htm}}.}
\item{Le Python Tutorial\footnote{\url{http://www.python.org/doc/current/tut/tut.html}} montre comment utiliser des listes comme des piles ou des files\footnote{\url{http://www.python.org/doc/current/tut/node7.html\#SECTION007110000000000000000}}.}
\item{La Python Knowledge Base\footnote{\url{http://www.faqts.com/knowledge-base/index.phtml/fid/199/}} répond aux questions courantes à propos des listes\footnote{\url{http://www.faqts.com/knowledge-base/index.phtml/fid/534}} et fourni de nombreux exemples de code utilisant des listes\footnote{\url{http://www.faqts.com/knowledge-base/index.phtml/fid/540}}.}
\item{La Python Library Reference\footnote{\url{http://www.python.org/doc/current/lib/}} résume toutes les méthodes des listes\footnote{\url{http://www.python.org/doc/current/lib/typesseq-mutable.html}}.}
\end{itemize}


\section{Présentation des tuples}

Un tuple (n-uplet) est une liste non-mutable. Une fois créé, un tuple ne peut en
aucune manière être modifié.

\begin{example}[Définition d'un tuple]
\begin{lstlisting}
>>> t = ("a", "b", "mpilgrim", "z", "example") (1)
>>> t
('a', 'b', 'mpilgrim', 'z', 'example')
>>> t[0]                                       (2)
'a'
>>> t[-1]                                      (3)
'example'
>>> t[1:3]                                     (4)
('b', 'mpilgrim')
\end{lstlisting}
\end{example}

\begin{enumerate}
\item{Un tuple est défini de la même manière qu'une liste sauf que l'ensemble d'éléments est entouré de parenthèses plutôt que de crochets.}
\item{Les éléments d'un tuple ont un ordre défini, tout comme ceux d'une liste.  Les indices de tuples débutent à zéro, tout comme ceux d'une liste, le  premier élément d'un tuple non vide est toujours \emph{t[0]}.Les indices négatifs comptent à partir du dernier élément du tuple, tout  comme pour une liste.}
\item{Le découpage fonctionne aussi, tout comme pour une liste. Notez que lorsque  vous découpez une liste, vous obtenez une nouvelle liste, lorsque vous découpez un tuple, vous obtenez un nouveau tuple.}
\end{enumerate}

\begin{example}[Les tuples n'ont pas de méthodes]
\begin{lstlisting}
>>> t
('a', 'b', 'mpilgrim', 'z', 'example')
>>> t.append("new")    (1)
Traceback (innermost last):
  File "<interactive input>", line 1, in ?
AttributeError: 'tuple' object has no attribute 'append'
>>> t.remove("z")      (2)
Traceback (innermost last):
  File "<interactive input>", line 1, in ?
AttributeError: 'tuple' object has no attribute 'remove'
>>> t.index("example") (3)
Traceback (innermost last):
  File "<interactive input>", line 1, in ?
AttributeError: 'tuple' object has no attribute 'index'
>>> "z" in t           (4)
True
\end{lstlisting}
\end{example}

\begin{enumerate}
\item{Vous ne pouvez pas ajouter d'élément à un tuple. Les tuples n'ont pas de méthodes \emph{append} ou \emph{extend}.}
\item{Vous ne pouvez pas enlever d'éléments d'un tuple. Les tuples n'ont pas de méthodes \emph{remove} ou \emph{pop}.}
\item{Vous ne pouvez pas rechercher d'éléments dans un tuple. Les tuples n'ont pas de méthode \emph{index}.}
\item{Vous pouvez toutefois utiliser \emph{in} pour vérifier l'existence d'un élément dans un tuple.}
\end{enumerate}

Mais à quoi servent donc les tuples ?

\begin{itemize}
\item {Les tuples sont plus rapides que les listes. Si vous définissez un ensemble constant de valeurs et que tout ce que vous allez faire est le parcourir, utilisez un tuple au lieu d'une liste.}
\item {Votre code est plus sûr si vous «~protégez en écriture~» les données qui n'ont pas besoin d'être modifiées. Utiliser un tuple à la place d'une liste revient à avoir une assertion implicite que les données sont constantes et que des mesures spécifiques sont nécéssaires pour modifier cette définition.}
\item{Vous vous souvenez que j'avais dit que que les clés de dictionnaire pouvaient être des entiers, des chaînes et «~quelques autres types~» ? Les tuples sont un de ces types. Ils peuvent être utilisé comme clé dans un dictionnaire, ce qui n'est pas le cas des listes. En fait, c'est plus compliqué que ça. Les clés de dictionnaire doivent être non-mutables. Les tuples sont non-mutables mais si vous avez un tuple contenant des listes, il est considéré comme mutable et n'est pas utilisable comme clé de dictionnaire. Seuls les tuples de chaînes, de nombres ou d'autres tuples utilisable comme clé peuvent être utilisés comme clé de dictionnaire.}
\item{Les tuples sont utilisés pour le formatage de chaînes, comme nous le verrons bientôt.}
\end{itemize}

\note{De tuple à liste à tuple}{
Les tuples peuvent être convertis en listes et vice-versa. La fonction prédéfinie \emph{tuple} prend une liste et retourne un tuple contenant les mêmes éléments et la fonction list prend un tuple et retourne une liste. En fait, \emph{tuple} gèle une liste et \emph{list} dégèle un tuple.}

\paragraph{Pour en savoir plus sur les tuples}
\begin{itemize}
\item{How to Think Like a Computer Scientist\footnote{\url{http://www.ibiblio.org/obp/thinkCSpy/}} explique les tuples et montre comment concaténer des tuples\footnote{\url{http://www.ibiblio.org/obp/thinkCSpy/chap10.htm}}.}
\item{La Python Knowledge Base\footnote{\url{http://www.faqts.com/knowledge-base/index.phtml/fid/199/}} vous apprendra à trier un tuple\footnote{\url{http://www.faqts.com/knowledge-base/view.phtml/aid/4553/fid/587}}.}
\item{Le Python Tutorial\footnote{\url{http://www.python.org/doc/current/tut/tut.html}} explique comment définir un tuple avec un seul élément\footnote{\url{http://www.python.org/doc/current/tut/node7.html\#SECTION007300000000000000000}}.}
\end{itemize}

\section{Définitions de variables}

Maintenant que vous pensez tout savoir à propos des dictionnaires, des tuples et des listes (hum!), revenons à notre programme d'exemple du Chapitre 2, \emph{odbchelper.py}.

Python dispose de variables locales et globales comme la plupart des autres langages, mais il n'a pas de déclaration explicite des variables. Les variables viennent au monde en se voyant assigner une valeur et sont automatiquement détruites lorsqu'elles se retrouvent hors de portée.

\begin{example}[Définition de la variable myParams]
\begin{lstlisting}
if __name__ == "__main__":
    myParams = {"server":"mpilgrim", \
                "database":"master", \
                "uid":"sa", \
                "pwd":"secret" \
                }
\end{lstlisting}
\end{example}

Il y a plusieurs points intéressants ici. Tout d'abord, notez l'indentation. Une instruction \emph{if} est un bloc de code et nécessite d'être indenté tout comme une fonction.

Deuxièmement, l'assignation de variable est une commande étalée sur plusieurs lignes avec une barre oblique («~\textbackslash~») servant de marque de continuation de ligne.

\note{Commandes multilignes}{
Lorsqu'une commande est étalée sur plusieurs lignes avec le marqueur de continuation de ligne («~\textbackslash~»), les lignes suivantes peuvent être indentées de n'importe qu'elle manière, les règles d'indentation strictes habituellement utilisées en Python ne s'appliquent pas. Si votre IDE Python indente automatiquement les lignes continuées, vous devriez accepter ses réglages par défauts sauf raison impérative.}

Les expressions entre parenthèses, crochets ou accolades (comme la définition d'un dictionnaire) peuvent être réparties sur plusieurs lignes avec ou sans le caractère de continuation («~\textbackslash~»). Je préfère inclure la barre oblique même
lorsqu'elle n'est pas requise, car je pense que cela rends le code plus lisible, mais c'est une question de style.

Troisièmement, vous n'avez jamais déclaré la variable \emph{myParams}, vous lui avez simplement assigné une valeur. C'est comme en VBScript sans l'option option \emph{explicit}. Heureusement, à l'inverse de VBScript, Python ne permet pas de référencer une variable à laquelle aucune valeur n'a été assigné. Tenter de le faire déclenchera une exception.

\subsection{Référencer des variables}

\begin{example}[Référencer une variable non assignée]
\begin{lstlisting}
>>> x
Traceback (innermost last):
  File "<interactive input>", line 1, in ?
NameError: There is no variable named 'x'
>>> x = 1
>>> x
1
\end{lstlisting}
\end{example}

Un jour, vous remercierez Python pour ça.

\subsection{Assignation simultanée de plusieurs valeurs}

Un des raccourcis les plus réjouissants existant en Python est l'utilisation de séquences pour assigner plusieurs valeurs en une fois.

\begin{example}[Assignation simultanée de plusieurs valeurs]
\begin{lstlisting}
>>> v = ('a', 'b', 'e')
>>> (x, y, z) = v     (1)
>>> x
'a'
>>> y
'b'
>>> z
'e'
\end{lstlisting}
\end{example}

\begin{enumerate}
\item {\emph{v} est un tuple de trois éléments et \emph{(x, y, z)} est un tuple de trois variables. Le fait d'assigner l'un à l'autre assigne chacune des valeurs de v a chacune des variables, dans leur ordre respectif.}
\end{enumerate}

Ce type d'assignation a de multiples usages. Je souhaite souvent assigner des noms a une série de valeurs. En C, vous utiliseriez \emph{enum} et vous listeriez manuellement chaque constante et la valeur associée, ce qui semble particulièrement fastidieux lorsque les valeurs sont consécutives. En Python, vous pouvez utiliser la fonction prédéfinie range avec l'assignation multiple de variables pour assigner rapidement des valeurs consécutives.

\begin{example}[Assignation de valeurs consécutives]
\begin{lstlisting}
>>> range(7)                                                                    (1)
[0, 1, 2, 3, 4, 5, 6]
>>> (MONDAY, TUESDAY, WEDNESDAY, THURSDAY, FRIDAY, SATURDAY, SUNDAY) = range(7)(2)
>>> MONDAY                                                                      (3)
0
>>> TUESDAY
1
>>> SUNDAY
6
\end{lstlisting}
\end{example}

\begin{enumerate}
\item{La fonction prédéfinie \emph{range} retourne une liste d'entiers. Dans sa forme la plus simple, elle prend une borne supérieure et retourne une séquence démarrant à 0, mais n'incluant pas la borne supérieure. (Si vous le souhaitez, vous pouvez spécifier une borne inférieure différente de 0 ou un pas d'incrément différent de 1. Vous pouvez faire un \emph{print range.\_\_doc\_\_} pour de plus amples détails.)}
\item{MONDAY, TUESDAY, WEDNESDAY, THURSDAY, FRIDAY, SATURDAY et SUNDAY sont les variables que nous définissons (cet exemple provient du module \emph{calendar}, qui est un petit module amusant qui affiche des calendriers comme le programme cal sous UNIX. Le module \emph{calendar} définit des constantes entières pour les jours de la semaine).}
\item{À présent, chaque variable possède sa valeur : MONDAY vaut 0, TUESDAY vaut 1 et ainsi de suite.}
\end{enumerate}

Vous pouvez aussi utiliser l'assignation multiple pour créer des fonctions qui retournent plusieurs valeurs, simplement en retournant un tuple contenant ces valeurs. L'appelant peut le traiter en tant que tuple ou assigner les valeurs à différentes variables. Beaucoup de bibliothèques standard de Python font cela, y compris le module os dont nous traiterons au Chapitre 6.

\paragraph{Pour en savoir plus sur les variables}
\begin{itemize}
\item{Le Python Reference Manual\footnote{\url{http://www.python.org/doc/current/ref/}} présente des exemples des cas où vous pouvez omettre le marqueur de continuation\footnote{\url{http://www.python.org/doc/current/ref/implicit-joining.html}} et où vous devez l'utiliser\footnote{\url{http://www.python.org/doc/current/ref/explicit-joining.html}}.}
\item{How to Think Like a Computer Scientist\footnote{\url{http://www.ibiblio.org/obp/thinkCSpy/}} montre comment utiliser l'assignation multiple pour échanger les valeurs de deux variables\footnote{\url{http://www.ibiblio.org/obp/thinkCSpy/chap09.htm}}.}
\end{itemize}

\section{Formatage de chaînes}

Python supporte le formatage de valeurs en chaînes de caractères. Bien que cela peut comprendre des expression très compliquées, l'usage le plus simple consiste à insérer des valeurs dans des chaînes à l'aide de marques \emph{\%s}.

\note{Python vs. C : formatage de chaînes}{
Le formatage de chaînes en Python utilise la même syntaxe que la fonction C \emph{sprintf}.}

\begin{example}[Présentation du formatage de chaînes]
\begin{lstlisting}
>>> k = "uid"
>>> v = "sa"
>>> "%s=%s" % (k, v) (1)
'uid=sa'
\end{lstlisting}
\end{example}

\begin{enumerate}
\item L'expression entière est évaluée en chaîne. Le premier \emph{\%s} est remplacé par la valeur de k, le second \emph{\%s} est remplacé par la valeur de \emph{v}. Tous les autres caractères de la chaîne (le signe d'égalité dans le cas présent) restent tels quels.
\end{enumerate}

Notez que \emph{(k, v)} est un tuple. Je vous avais dit qu'ils servaient à quelque chose.

Vous pourriez pensez que cela représente beaucoup d'efforts pour le formatage de chaîne se bornait à la concaténation. Il n'y est pas question uniquement de formatage mais également de conversion de types.

\begin{example}[Formatage de chaîne et concaténation]
\begin{lstlisting}
>>> uid = "sa"
>>> pwd = "secret"
>>> print pwd + " is not a good password for " + uid      (1)
secret is not a good password for sa
>>> print "%s is not a good password for %s" % (pwd, uid) (2)
secret is not a good password for sa
>>> userCount = 6
>>> print "Users connected: %d" % (userCount, )           (3) (4)
Users connected: 6
>>> print "Users connected: " + userCount                 (5)
Traceback (innermost last):
  File "<interactive input>", line 1, in ?
TypeError: cannot concatenate 'str' and 'int' objects
\end{lstlisting}
\end{example}

\begin{enumerate}
\item + est l'opérateur de concaténation de chaînes.
\item Dans ce cas trivial, le formatage de chaînes mène au même résultat que la
    concaténation.
\item \emph{(userCount, )} est un tuple contenant un seul élément. Oui, la syntaxe est
    un peu étrange mais il y a un excellente raison : c'est un tuple sans aucune
    ambiguité. En fait, vous pouvez toujours mettre une virgule après
    l'élément terminal lors de la définition d'une liste, d'un tuple ou d'un
    dictionnaire mais cette virgule est obligatoire lors de la définition d'un
    tuple avec un élément unique. Si ce n'était pas le cas, Python ne pourrait
    distinguer si \emph{(userCount)} est un tuple avec un seul élément ou juste la
    valeur \emph{userCount}.
\item Le formatage de chaîne fonctionne également avec des entiers en spécifiant
    \emph{\%d} au lieu de \emph{\%s}.
\item Si vous tentez de concaténer une chaîne avec un autre type, Python va
    déclencher une exception. Au contraire du formatage de chaîne, la
    concaténation ne fonctionne que si tout les objets sont déjà de type
    chaîne.
\end{enumerate}

Comme la fonction printf en C, le formatage de chaînes en Python est un véritable couteau suisse. Il y a des options à profusion et des modificateurs de format spécifiques pour de nombreux types de valeurs.

\begin{example}[Formatage de nombres]
\begin{lstlisting}
>>> print "Today's stock price: %f" % 50.4625   (1)
50.462500
>>> print "Today's stock price: %.2f" % 50.4625 (2)
50.46
>>> print "Change since yesterday: %+.2f" % 1.5 (3)
+1.50
\end{lstlisting}
\end{example}

\begin{enumerate}
\item L'option de formatage \emph{\%f} considère la valeur comme un nombre décimal et
    l'affiche avec six chiffres après la virgule.
\item Le modificateurs ".2" de l'option \emph{\%f} tronque la valeur à deux chiffres
    après la virgule.
\item On peut également combiner les modificateurs. Ajouter le modificateur +
    affiche le signe positif ou négatif avant la valeur. Notez que le
    modificateur ".2" est toujours en place et qu'il formate la valeur avec
    exactement deux chiffres après la virgule.
\end{enumerate}

\paragraph{Pour en savoir plus sur le formatage de liste}
\begin{itemize}
\item{La Python Library Reference\footnote{\url{http://www.python.org/doc/current/lib/}} résume
    tous les caractères spéciaux utilisés pour le formatage de chaînes\footnote{\url{http://www.python.org/doc/current/lib/typesseq-strings.html}}.}
\item{Effective AWK Programming \footnote{\url{http://www-gnats.gnu.org:8080/cgi-bin/info2www?(gawk)Top}} explique tous les caractères de formatage\footnote{\url{http://www-gnats.gnu.org:8080/cgi-bin/info2www?(gawk)Control+Letters}} et des technique de formatage avancées comme le réglage de la largeur ou de la précision et le remplissage avec des zéros\footnote{\url{http://www-gnats.gnu.org:8080/cgi-bin/info2www?(gawk)Format+Modifiers}}.}
\end{itemize}

\section{Mutation de listes}

Une des caractéristiques les plus puissantes de Python est la \emph{list comprehension} (création fonctionnelle de listes) qui fournit un moyen concis d'appliquer une fonction sur chaque élément d'une liste afin d'en produire une nouvelle.

\begin{example}[Présentation des \emph{list comprehensions}]
\begin{lstlisting}
>>> li = [1, 9, 8, 4]
>>> [elem*2 for elem in li]      (1)
[2, 18, 16, 8]
>>> li                           (2)
[1, 9, 8, 4]
>>> li = [elem*2 for elem in li] (3)
>>> li
[2, 18, 16, 8]
\end{lstlisting}
\end{example}

\begin{enumerate}
\item Pour comprendre cette ligne, observez là de droite à gauche. \emph{li} est la
    liste que vous appliquez. Python la parcourt un élément à la fois, en
    assignant temporairement la valeur de chacun des éléments à la variable
    elem. Python applique ensuite la fonction \emph{elem*2} et ajoute le résultat à la
    liste retournée.
\item Notez que les \emph{list comprehensions} ne modifient pas la liste initiale.
\item Vous pouvez assigner le résultat d'une \emph{list comprehension} à la variable que
    vous traitez. Python assemble la nouvelle liste en mémoire et assigne le
    résultat à la variable une fois la transformation terminée.
\end{enumerate}

Voici les \emph{list comprehensions} dans la fonction \emph{buildConnectionString} que nous
avons déclaré au Chapitre 2 :

\begin{lstlisting}
["%s=%s" % (k, v) for k, v in params.items()]
\end{lstlisting}

Notez tout d'abord que vous appelez la fonction \emph{items} du dictionnaire \emph{params}. Cette fonction retourne une liste de tuples avec toutes les données stockées dans le dictionnaire.

\begin{example}[Les fonctions keys, values et items]
\begin{lstlisting}
>>> params = {"server":"mpilgrim", "database":"master", "uid":"sa", "pwd":"secret"}
>>> params.keys()   (1)
['server', 'uid', 'database', 'pwd']
>>> params.values() (2)
['mpilgrim', 'sa', 'master', 'secret']
>>> params.items()  (3)
[('server', 'mpilgrim'), ('uid', 'sa'), ('database', 'master'), ('pwd', 'secret')]
\end{lstlisting}
\end{example}

\begin{enumerate}
\item La méthode \emph{keys} d'un dictionnaire retourne la liste de toutes les clés.
    Cette liste ne suit pas l'ordre dans lequel le dictionnaire a été défini
    (souvenez-vous, les éléments d'un dictionnaire ne sont pas ordonnés) mais
    cela reste une liste.
\item La méthode \emph{values} retourne la liste de toutes les valeurs. La liste est
    dans le même ordre que celle retournée parkeys, on a donc \emph{params.values()[n] == params[params.keys[n]]} pour toute valeur de \emph{n}.
\item La méthode \emph{items} retourne une liste de tuples de la forme (clé, valeur). La liste contient toutes les données stockées dans le dictionnaire.
\end{enumerate}

Voyons maintenant ce que fait \emph{buildConnectionString}. Elle prend une liste,
\emph{param.items()}, et crée une nouvelle liste en appliquant une instruction de
formatage de chaîne à chacun de ses éléments. La nouvelle liste aura le même
nombre d'éléments que \emph{params.items()} mais chaque élément sera une chaîne qui
contient à la fois une clé et la valeur qui lui est associée dans le dictionnaire \emph{params}.

\begin{example}[\emph{List comprehensions} dans \emph{buildConnectionString}, pas à pas]
\begin{lstlisting}
>>> params = {"server":"mpilgrim", "database":"master", "uid":"sa", "pwd":"secret"}
>>> params.items()
[('server', 'mpilgrim'), ('uid', 'sa'), ('database', 'master'), ('pwd', 'secret')]
>>> [k for k, v in params.items()]                (1)
['server', 'uid', 'database', 'pwd']
>>> [v for k, v in params.items()]                (2)
['mpilgrim', 'sa', 'master', 'secret']
>>> ["%s=%s" % (k, v) for k, v in params.items()] (3)
['server=mpilgrim', 'uid=sa', 'database=master', 'pwd=secret']
\end{lstlisting}
\end{example}

\begin{enumerate}
\item Notez que nous utilisons deux variables pour parcourir la liste
    \emph{params.items()}. Il s'agit d'un autre usage de l'assignment multiple. Le
    premier élément de \emph{params.items()} est \emph{('server', 'mpilgrim')}, donc lors de
    la première itération de la transformation, \emph{k} va prendre la valeur \emph{'server'}
    et \emph{v} la valeur \emph{'mpilgrim'}. Dans ce cas, nous ignorons la valeur de \emph{v} et
    plaçons uniquement la valeur de \emph{k} dans la liste résultante. Cette
    transformation correspond donc au comportement de \emph{params.keys()}. (Vous
    n'utiliseriez pas réellement une \emph{list comprehension} comme ceci dans du vrai
    code; il s'agit d'un exemple exagérément simple pour que vous compreniez ce
    qui se passe.)
\item Nous faisons la même chose ici, mais nous ignorons la valeur de \emph{k} de telle
    sorte que le résultat est équivalent à celui de \emph{params.values()}.
\item En combinant les deux exemples précédent avec le formatage de chaîne, nous
    obtenons une liste de chaînes comprenant la clé et la valeur de chaque
    élément du dictionnaire. Cela ressemble étonnamment à la sortie du
    programme, tout ce qui reste à faire maintenant est la jointure des
    éléments de cette liste en une seule chaîne.
\end{enumerate}

\paragraph{Pour en savoir plus sur les \emph{list comprehensions}}
\begin{itemize}
\item Le Python Tutorial\footnote{\url{http://www.python.org/doc/current/tut/tut.html}} traite d'une autre manière de transformer des listes avec la fonction prédéfinie \emph{map}\footnote{\url{http://www.python.org/doc/current/tut/node7.html\#SECTION007130000000000000000}}.
\item Le Python Tutorial\footnote{\url{http://www.python.org/doc/current/tut/tut.html}} montre comment emboîter des mutations de listes\footnote{\url{http://www.python.org/doc/current/tut/node7.html\#SECTION007140000000000000000}}.
\end{itemize}

\section{Jointure de listes et découpage de chaînes}

Nous avons une liste de paires clé-valeur sous la forme \emph{clé=valeur} et nous voulons les assembler au sein d'une même chaîne. Pour joindre une liste de chaînes en une seule, nous pouvons utiliser la méthode \emph{join} d'un objet chaîne.

Voici un exemple de jointure de liste provenant de la fonction \emph{buildConnectionString} :

\begin{lstlisting}
return ";".join(["%s=%s" % (k, v) for k, v in params.items()])
\end{lstlisting}

Une remarque intéressante avant de continuer. Je ne cesse de répéter que les fonctions sont des objets, que les chaînes sont des objets, que tout est objet. Vous pourriez penser que seules les variables de type chaîne sont des objets.
Mais ce n'est pas le cas, regardez de plus près cet exemple et vous verrez que la chaîne « ; » est elle même un objet dont vous appelez la méthode \emph{join}.

La méthode \emph{join} assemble les éléments d'une liste pour former une chaîne unique, chaque élément étant séparé par un point virgule. Le séparateur n'est pas forcément un point-virgule, il n'est même pas forcément un caractère
unique. Il peut être n'importe quelle chaîne.

\attention{Attention}{
La méthode \emph{join} ne fonctionne qu'avec des listes de chaînes; elle n'applique pas la conversion de types. La jointure d'une liste comprenant au moins un élément non-chaîne déclenchera une exception.}

\begin{example}[Sortie de \emph{odbchelper.py}]
\begin{lstlisting}
>>> params = {"server":"mpilgrim", "database":"master", "uid":"sa", "pwd":"secret"}
>>> ["%s=%s" % (k, v) for k, v in params.items()]
['server=mpilgrim', 'uid=sa', 'database=master', 'pwd=secret']
>>> ";".join(["%s=%s" % (k, v) for k, v in params.items()])
'server=mpilgrim;uid=sa;database=master;pwd=secret'
\end{lstlisting}
\end{example}

La chaîne est alors retournée de la fonction \emph{odbchelper} et affichée par le bloc appelant, ce qui vous donne la sortie qui vous a tant émerveillé quand vous avez débuté la lecture de ce chapitre.

Vous vous demandez probablement s'il existe une méthode analogue permettant de découper une chaîne en liste. Et bien sur elle existe, elle porte le nom de \emph{split}.

\begin{example}[Découpage d'une chaîne]
\begin{lstlisting}
>>> li = ['server=mpilgrim', 'uid=sa', 'database=master', 'pwd=secret']
>>> s = ";".join(li)
>>> s
'server=mpilgrim;uid=sa;database=master;pwd=secret'
>>> s.split(";")    (1)
['server=mpilgrim', 'uid=sa', 'database=master', 'pwd=secret']
>>> s.split(";", 1) (2)
['server=mpilgrim', 'uid=sa;database=master;pwd=secret']
\end{lstlisting}
\end{example}

\begin{enumerate}
\item \emph{split} fait l'inverse de join en découpant une chaîne en une liste de
    plusieurs éléments. Notez que le délimiteur («~;~») est totalement
    supprimé, il n'apparaît dans aucun des éléments de la liste retournée.
\item \emph{split} prend en deuxième argument optionnel le nombre de découpages à
    effectuer («~Des arguments optionnels ?~» Vous apprendrez à en définir dans
    vos propres fonctions au prochain chapitre.)
\end{enumerate}

\note{Rechercher avec split}{
La commande \emph{une\_chaîne.split(delimiteur, 1)} est une technique utile pour chercher une sous-chaîne dans une chaîne et utiliser tout ce qui précède cette sous-chaîne (le premier élément de la liste retournée) et tout ce qui la suit (le second élément de la liste retournée).}

\paragraph{Pour en savoir plus sur les méthodes de chaînes}
\begin{itemize}
\item{La Python Knowledge Base\footnote{\url{http://www.faqts.com/knowledge-base/index.phtml/fid/199/}} répond aux questions courantes à propose des chaînes\footnote{\url{http://www.faqts.com/knowledge-base/index.phtml/fid/480}} et dispose de nombreux exemples de code utilisant des chaînes\footnote{\url{http://www.faqts.com/knowledge-base/index.phtml/fid/539}}.}
\item{La Python Library Reference\footnote{\url{http://www.python.org/doc/current/lib/}} récapitule toutes les méthodes de chaînes\footnote{\url{http://www.python.org/doc/current/lib/string-methods.html}}.}
\item{La Python Library Reference\footnote{\url{http://www.python.org/doc/current/lib/}} documente le module \emph{string}\footnote{\url{http://www.python.org/doc/current/lib/module-string.html}}.}
\item{The Whole Python FAQ\footnote{\url{http://www.python.org/doc/FAQ.html}} explique pourquoi \emph{join} est une méthode de chaînes\footnote{\url{http://www.python.org/cgi-bin/faqw.py?query=4.96&querytype=simple&casefold=yes&req=search)}}et non une méthode de liste.}
\end{itemize}

\subsection{Note historique sur les méthodes de chaînes}

Lorsque j'ai débuté l'apprentissage de Python, je m'attendais à ce que \emph{join} soit une méthode de liste qui aurait pris un séparateur comme argument. Beaucoup de gens pensent la même chose et il y a une véritable histoire derrière la méthode \emph{join}. Avant Python 1.6, les chaînes n'étaient pas pourvue de toutes ces méthodes si utiles. Il y avait un module string séparé qui contenait toutes les fonctions de manipulation de chaînes de caractères, chacune prenant une chaîne comme premier argument. Ces fonctions ont été considérées assez importantes pour être intégrées dans les chaînes elles même, ce qui semblait logique pour des fonctions comme \emph{lower}, \emph{upper} et \emph{split}. Mais beaucoup de programmeurs Python issus du noyau dur émirent des objections quant à la méthode join en arguant du fait qu'elle devrait être plutôt une méthode de liste ou tout simplement rester une fonction du module string (qui contient encore bien des choses utiles). J'utilise exclusivement la nouvelle méthode \emph{join} mais vous verrez du code écrit des deux façons et si cela vous pose un réel problème, vous pouvez toujours opter pour l'ancienne fonction \emph{string.join}.

\section{Résumé}\label{Résumé}

À présent, le programme \emph{odbchelper.py} et sa sortie devraient vous paraître
parfaitement clairs.

\begin{lstlisting}
def buildConnectionString(params):
    """Build a connection string from a dictionary of parameters.

    Returns string."""
    return ";".join(["%s=%s" % (k, v) for k, v in params.items()])

if __name__ == "__main__":
    myParams = {"server":"mpilgrim", \
                "database":"master", \
                "uid":"sa", \
                "pwd":"secret" \
                }
    print buildConnectionString(myParams)
\end{lstlisting}

Voici la sortie de \emph{odbchelper.py} :
\begin{lstlisting}[style=none]
server=mpilgrim;uid=sa;database=master;pwd=secret
\end{lstlisting}

Avant de vous plonger dans le chapitre suivant, assurez vous que vous vous
sentez à l'aise pour :
\begin{itemize}
\item{utiliser l'IDE Python pour tester des expressions de manière interactive ;}
\item{écrire des modules Python et les exécuter depuis votre IDE ou en ligne de commande ;}
\item{importer des modules et appeler leurs fonctions ;}
\item{déclarer des fonctions et utiliser des doc string, des variables locales et une indentation correcte ;}
\item{définir des dictionnaires, des tuples et des listes ;}
\item{accéder aux attributs et méthodes de tout objet, y compris les chaînes, les listes, les dictionnaires, les fonctions et les modules ;}
\item{concaténer des valeur avec le formatage de chaînes ;}
\item{utiliser les \emph{lists comprehensions} pour la mutation de listes ;}
\item{découper des chaînes en listes et joindre des listes en chaînes.}
\end{itemize}

%\include{chapitres/DIP_chapitre}
%\include{chapitres/DIP_chapitre}
%\include{chapitres/DIP_chapitre}
%\include{chapitres/DIP_chapitre}
%\include{chapitres/DIP_chapitre}

\end{document}


